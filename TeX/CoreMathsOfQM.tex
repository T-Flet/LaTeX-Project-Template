\section{Core Mathematics of Quantum Mechanics}
\label{sec:CoreMathsOfQM}



\subsection{Mathematical Bases}
\label{sec:SubMathBases}

The content of this appendix only serves to showcase a missing reference, how Appendix references are shown and simple lists.\\

As will be made clear shortly (in Definition \ref{def:QMPostulates}), it is specifically \gls{Unitary} Hilbert Spaces which are of concern to \gls{QM},
and the common notation in treating them is different from the standard mathematical one used for Inner Product Space:

\begin{defn}\label{def:DiracNotation}
    The \textbf{Dirac Notation} (or \textbf{Bra-Ket Notation}) is the following:
    \begin{itemize}[noitemsep]
        \item Vectors are denoted $|v\rangle$ (``Ket'')
        \item Covectors (vectors of the \gls{Dual Space}) are denoted $\langle v|$ (``Bra'')
        \item Inner Products are denoted $\langle\cdot|\cdot\rangle$ (``Bra-(c)-Ket''), where \textbf{linearity is swapped} with respect to standard notation:
            here it is the second term which is linear, while the first is conjugate-linear
        \item Outer Products are denoted $|\cdot\rangle\langle\cdot|$ (for matching `Ket' and `Bra' this is simply a projector operator)
    \end{itemize}
    The fact that `Bra's $\langle\psi|$ are precisely the Dual Vectors of `Ket's $|\psi\rangle$ is the key to the usefulness of this notation:
    \begin{itemize}[noitemsep]
        \item The relation between the two is the very elegant one of \glspl{Hermitian Conjugate}: $\langle\psi| = |\psi\rangle^{\dagger}$
        \item Making the `Bra's interpretable as linear functions on Kets justifies the notation for the Inner Product: $\langle v| \cdot |w\rangle$ is indeed
            equivalent to $\langle v|w \rangle$ (which is equal to $\langle |w\rangle,|v\rangle \rangle$ in standard Inner Product notation with Ket notation vectors)
    \end{itemize}
\end{defn}


