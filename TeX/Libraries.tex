\section{Libraries}
\label{sec:Libraries}

This section is here to showcase the listings environment varieties:
inline, as item label in a description environment and as a stand-alone listing. 



\subsection{GSM}
\label{sec:Sub_GSM}

The \acrlong{GSM} library \cite{GSM} implements a computational construct similar to a Turing Machine over a graph,
where states are node combinations (though more information may be stored) and where the arbitrary transition function
can update both state and graph.

\begin{defn}\label{def:GSM}
Given a \lstinline{Graph} with typed nodes and a \lstinline{State} object of arbitrary structure,
a \textbf{\gls{GSM}} is defined by the functions it applies to perform a step:
\begin{description}[noitemsep]
    \lstitem{Selector} A function to extract a list of nodes from the arbitrarily-structured \lstinline{State} 
    \lstitem{Scanner} A generalised neighbourhood function, which scans the graph ``around'' the state nodes, 
            optionally applying some filter (e.g. node type), and returns a scored list of nodes for further processing
    \lstitem{Updater} A function to process the scan result and thus update the state and/or the graph itself
\end{description}
\end{defn}



\subsection{GPy-ABCD}
\label{sec:Sub_GPyABCD}

A full listing, which has a label, a caption and language-aware highlighting.

\begin{lstlisting}[label=lst:GPyABCD_function, language=Python, caption=Main GPy-ABCD \cite{GPy_ABCD} model-space search function with default arguments, breaklines=true, basicstyle=\ttfamily\small]
    import numpy as np
    from GPy_ABCD import *


    if __name__ == '__main__':
        # Example data
        X = np.linspace(-10, 10, 101)[:, None]
        Y = np.cos((X - 5) / 2) ** 2 * X * 2 + np.random.randn(101, 1)

        # Main function call with default arguments
        best_mods, all_mods, all_exprs, expanded, not_expanded = explore_model_space(X, Y,
            start_kernels = start_kernels['Default'], p_rules = production_rules['Default'],
            utility_function = BIC, rounds = 2, beam = [3, 2, 1], restarts = 5,
            model_list_fitter = fit_mods_parallel_processes, optimiser = GPy_optimisers[0],
            verbose = True)

        print('\nFull lists of models by round:')
        for mod_depth in all_mods: print(', '.join([str(mod.kernel_expression) for mod in mod_depth]) + f'\n{len(mod_depth)}')

        print('\n\nTop-3 models\' details:')
        for bm in best_mods[:3]:
            model_printout(bm) # See the definition of this convenience function for examples of model details' extraction
            print('Prediction at X = 11:', bm.predict(np.array([11])[:, None]), '\n')

        from matplotlib import pyplot as plt
        plt.show()
\end{lstlisting}


