%%%% ACRONYMS %%%%%%%%%%%%%%%%%%%%%%%%%%%%%%%%%%%%%%%%%%%%%%%%%%%%%%%%%%%%%%%%%%%%%%%%%%%%%%%%%%%%%

\newacronym{QM}{QM}{Quantum Mechanics}
\newacronym{QI}{QI}{Quantum Information}

\newacronym{GSM}{GSM}{Graph State Machine}



%%%% BASIC DEFINITIONS %%%%%%%%%%%%%%%%%%%%%%%%%%%%%%%%%%%%%%%%%%%%%%%%%%%%%%%%%%%%%%%%%%%%%%%%%%%%

%% Algebra

\begin{comment}

\newglossaryentry{Group}{
    name={Group},
    description={
        A Group $(G,\cdot)$ is an algebraic structure consisting of a set $G$ with an operation $\cdot$ acting on it satisfying the axioms:
        \begin{description}[noitemsep]
            \item [G1: Associativity] $\cdot$ has to be associative: $\forall \, x,y,z \in G, \, (xy)z = x(yz)$ 
            \item [G2: Identity] There has to exist an identity element: $\exists \, 1_{G} \in G \, \big| \, \forall \, x \in G, 1_{G}x = x = x1_{G}$
            \item [G3: Inverses] Each element has to have an inverse: $\forall \, x \in G, \exists \, x^{-1} \in G \, \big| \, xx^{-1} = 1_{G} = x^{-1}x$
        \end{description}
    }
}

\newglossaryentry{Abelian}{name={Abelian}, description={(Of an algebraic structure) commutative under an operation, e.g. abelian \glspl{Group} and \glspl{Ring} have commutative $\cdot \,$ }}
\newglossaryentry{Monoid}{name={Monoid}, description={A Monoid $(M,\cdot)$ is an algebraic structure satisfying the \textbf{G1} and \textbf{G2} \gls{Group} axioms}}

\newglossaryentry{Ring}{
    name={Ring},
    description={
        A Ring $(R,+,\cdot)$ is an algebraic structure consisting of a set $R$ with two operations $+$ and $\cdot$ acting on it satisfying the axioms:
        \begin{description}[noitemsep]
            \item [R1: Additive Group] $(R,+)$ has to be an \gls{Abelian} \gls{Group}; call its identity element $0_{R}$
            \item [R2: Multiplicative Monoid] $(R,\cdot)$ has to be a \gls{Monoid}; call its identity element $1_{R}$
            \item [R3: Multiplication Distributivity] $\cdot$ has to distribute over $+$: $\forall \, x,y,z \in R, \, x(y+z) = xy+xz \, \wedge \, (x+y)z = xz + yz$
        \end{description}
    }
}

\newglossaryentry{Division Ring}{name={Division Ring}, description={A \gls{Ring} in which every non-zero element has an inverse under $\cdot \,$}}
\newglossaryentry{Field}{name={Field}, description={A commutative \gls{Division Ring}, e.g. $\mathbb{R}$ or $\mathbb{C}$}}

\newglossaryentry{Vector Space}{
    name={Vector Space},
    description={
        A Vector Space $V$ over a \gls{Field} $F$ is an algebraic structure satisfying the axioms:
        \begin{description}[noitemsep]
            \item [V1: Additive Group] $(V,+)$ has to be an \gls{Abelian} \gls{Group}
            \item [V2: Scalar Multiplication] Define the commutative operation $F \times V \rightarrow V$ of scalar multiplication with $1_{F}$ as its identity element,
                    such that $\forall \, \bm{v} \in V, \, 1_{F}\bm{v} = \bm{v} = \bm{v}1_{F}$
            \item [V3: Distributivity] Scalar multiplication has to distribute over both vector and field additions:
                    $\forall \, \bm{v},\bm{w} \in V, \, x,y \in F, \, x(\bm{v}+\bm{w}) = x\bm{v}+x\bm{w} \, \wedge \, (x+y)\bm{v} = x\bm{v}+y\bm{v}$
        \end{description}
        Note: vectors are (1,0)-\glspl{Tensor} (while dual vectors are (0,1)-\glspl{Tensor})
    }
}

\end{comment}

\newglossaryentry{Dual Space}{
    name={Dual Space},
    description={
        The Dual Space $V^{*}$ of a Vector Space $V$ over a Field $F$ is the set of all linear maps $\phi: V \rightarrow F$ from the Vector Space to its underlying Field;
        this space becomes a Vector Space itself with the following operations defined on it $\forall \, \phi,\psi \in V^{*}, \, \bm{v} \in V, \, x \in F$:
        \begin{description}[noitemsep]
            \item [Addition] $(\phi+\psi)(\bm{v}) = \phi(\bm{v}) + \psi(\bm{v})$
            \item [Scalar Multiplication] $(x\phi)(\bm{v}) = x(\phi(\bm{v}))$
        \end{description}
    }
}

\newglossaryentry{Banach Space}{
    name={Banach Space},
    description={
        A Banach Space is a Normed Space $(V,||\cdot||)$ which is also a Complete Metric Space $(V,d)$ with metric
        $d$ naturally induced by norm $||\cdot||$: $d(\bm{v},\bm{w}) = ||\bm{v} - \bm{w}|| \; \forall \, \bm{v},\bm{w} \in V$
    }
}

\newglossaryentry{Hermitian Conjugate}{
    name={Hermitian Conjugate},
    description={
        The Hermitian Conjugate $M^{\dagger}$ of a matrix $M$ is the complex conjugate of the transpose of $M$:
        $M^{\dagger} = \overline{M^{T}} (= \overline{M}^{T}) \iff M^{\dagger}_{ij} = \overline{M_{ji}} \; \forall \, i,j$
        (This definition can be extended to multidimensional arrays and operators)
    }
}

\newglossaryentry{Hermitian}{
    name={Hermitian},
    description={
        A matrix (or operator) $M$ is Hermitian if it is equal to its \gls{Hermitian Conjugate}: $M = M^{\dagger}$
        Hermitian matrices (and operators) have real eigenvalues and Orthogonal associated eigenvectors.
        Sums and inverses of Hermitian matrices (and operators) are Hermitian.
        Products of Hermitian matrices (and operators) are Hermitian if and only if the matrices commute
    }
}

\newglossaryentry{Unitary}{
    name={Unitary},
    description={
        A finite-dimensional matrix (or operator) $M$ is unitary if its \gls{Hermitian Conjugate} is its inverse:
        $M^{\dagger} = M^{-1} \iff M^{\dagger} \cdot M = I = M \cdot M^{\dagger}$ \\
        In the infinite-dimensional case $M^{\dagger} \cdot M = I \centernot\implies M \cdot M^{\dagger} = I$,
        but sufficient conditions for unitarity are $M^{\dagger} \cdot M = I$ and surjectivity
    }
}
